\chapter{Conclusioni}

Il progetto ci ha permesso di comprendere quanto il web semantico rappresenti e rappresenterà in futuro uno strumento indispensabile in molteplici ambiti, per ottenere informazioni anche complesse con la semplicità di un interrogazione ad un sistema interconnesso, collaborativo ed interoperabile, superando tutte quelle che sono le attuali limitazioni del web.
\\\\
Molte informazioni ricavate tramite query effettuate su Motology stessa senza l'impiego del web semantico sarebbero state svariati minuti di ricerca ed integrazione di dati presenti su motori di ricerca tradizionali, riteniamo pertanto che l'obiettivo di partenza sia stato pienamente soddisfatto dal nostro sistema.

\section{Sviluppi futuri}
Gli sviluppi futuri del progetto sono numerosi e riguardano diversi campi: 

\begin{itemize}
\item Web semantico: l'ontologia può facilmente essere impiegata in altri progetti, sia di natura più generale che più dettagliata, inoltre potrebbe essere estesa a tutti gli sport motociclistici (superbike, enduro etc.), potrebbe essere integrata con dati dei veicoli in modo da ottenere informazioni più dettagliate riguardo i singoli componenti della moto etc.
\item Machine learning: un ontologia di questo tipo potrebbe essere utilizzata in progetti di machine learning per allenare modelli predittivi analizzando trend che possono emergere dallo studio dei dati interconnessi da Motology.

\end{itemize}